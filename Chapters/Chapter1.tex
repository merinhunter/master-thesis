% Chapter 1: Introduction

\chapter{Introducción} % Main chapter title

\label{Chapter1} % For referencing the chapter elsewhere, use \ref{Chapter1}

%----------------------------------------------------------------------------------------

% Define some commands to keep the formatting separated from the content
\newcommand{\keyword}[1]{\textbf{#1}}
\newcommand{\tabhead}[1]{\textbf{#1}}
\newcommand{\code}[1]{\texttt{#1}}
\newcommand{\file}[1]{\texttt{\bfseries#1}}
\newcommand{\option}[1]{\texttt{\itshape#1}}

%----------------------------------------------------------------------------------------

\section{Motivación}

Es de sobra conocida la importantica de la analítica de datos en nuestro tiempo, así también lo es la referida al desarrollo de \emph{software}\index{software}. Herramientas como GrimoireLab\index{GrimoireLab} posibilitan este fin.

Para acercar su uso a perfiles no técnicos, los creadores de GrimoireLab\index{GrimoireLab} idearon Cauldron\index{Cauldron}, un SaaS de GrimoireLab que funciona como una aplicación web. Con un par de clics, un usuario es capaz de poner en marcha el sistema y obtener datos referidos al desarrollo de \emph{software}\index{software} fácilmente.

Sin embargo, esta plataforma no está exenta de defectos. Como parte del equipo desarrollador de Cauldron\index{Cauldron}, el autor de este Trabajo de Fin de Máster conoce de primera mano los puntos de mejora del sistema.

Con este pretexto comienza el desarrollo de una herramienta que facilite la analítica de datos referente al desarrollo de \emph{software}\index{software}, soslayando las dificultades encontradas durante el desarrollo de Cauldron\index{Cauldron}.

%----------------------------------------------------------------------------------------

\section{Estructura de la memoria}

La estructura que esta memoria sigue es la siguiente:

\begin{itemize}
    \item En el Capítulo~\ref{Chapter2} se explican varios conceptos y tecnologías que ya existen y que se han usado para llevar a cabo el proyecto.
    \item En el Capítulo~\ref{Chapter3} se detalla el desarrollo del proyecto, desde el trabajo previo al comienzo del proyecto hasta los problemas encontrados durante el desarrollo, pasando por los diferentes prototipos de la aplicación.
    \item En el Capítulo~\ref{Chapter4} se exponen las especificaciones del sistema final, incluyendo instrucciones detalladas para poner la aplicación en funcionamiento, así como un caso de uso.
    \item En el Capítulo~\ref{Chapter5} se finaliza la memoria haciendo una reflexión sobre los resultados y las posibles líneas de desarrollo futuras.
\end{itemize}

%----------------------------------------------------------------------------------------

\section{Objetivos}

\subsection{Objetivo general}

El objetivo principal de este proyecto es el desarrollo de una plataforma similar a Cauldron\index{Cauldron}, que permita el análisis de ecosistemas de desarrollo de \emph{software}\index{software}, modificando algunas partes de su funcionamiento original, añadiendo nuevas características, y resolviendo varios problemas que el proyecto original llevaba arrastrando desde su comienzo.

\subsection{Objetivos específicos}

Para llevar a cabo el objetivo principal del proyecto, se han elaborado una serie de objetivos más específicos:

\begin{itemize}
    \item El esquema de funcionamiento en Cauldron\index{Cauldron} gira en torno a la generación de informes alterables, esto es, cuando un usuario analiza una o varias fuentes de datos es capaz de modificar (incluso cuando el análisis no ha finalizado) la lista de objetos a analizar. De la misma manera, un informe realizado hace meses continúa actualizando los datos mostrados si otro usuario (o el creador) lo solicita. Esta característica, si bien permitía no tener que analizar varias veces un mismo repositorio para tener datos actualizados, generaba bastantes problemas de desarrollo al tener que lidiar con múltiples factores que podían alterar los datos de un informe. Uno de los objetivos que persigue este proyecto es que la nueva plataforma genere informes inalterables, creando un flujo más simple desde la solicitud por parte del usuario al \emph{output} de la aplicación.
    \item Muchos de los servicios que componen Cauldron\index{Cauldron} tienen una fuerte dependencia entre ellos, de manera que el fallo en uno puede ocasionar el fallo general del sistema. En un entorno en el que el desarrollo \emph{serverless} y las arquitecturas basadas en microservicios son cada vez más frecuentes, es necesaria la creación de sistemas desacoplados que funcionen en su conjunto para dar un servicio. Este proyecto persigue el desarrollo de varios componentes \emph{software}\index{software} independientes que colaboren entre ellos para dar un servicio al usuario.
    \item Cauldron\index{Cauldron} utiliza GrimoireLab\index{GrimoireLab} (más concretamente la herramienta conocida como SirMordred) para realizar las tareas de recogida y enriquecimiento de datos. La forma en la que Cauldron opera con la herramienta es mediante un contenedor Docker\index{Docker} personalizado, el cual deriva de uno proporcionado por GrimoireLab, lo que añade más complejidad al incluir un componente extra que necesita ser mantenido. Este objetivo pretende simplificar el uso de GrimoireLab mediante la ejecución del contenedor Docker original de GrimoireLab. Aparte, otro de los componentes de GrimoireLab, llamado Sorting Hat, añade limitaciones al sistema (o complejidad, según se mire) al ser requisito inalterable para su funcionamiento el uso de una base de datos MariaDB. Debido a esto, y a que el RGPD\footnote{Reglamento General de Protección de Datos} limita el valor que se le puede dar a una plataforma como esta, se ha decidido eliminar este componente del esquema de la nueva plataforma.
    \item El último objetivo que busca este proyecto está relacionado con mejoras menores respecto a Cauldron\index{Cauldron}, como el uso de Poetry\index{Poetry} para la gestión de dependencias, la sustitución del servidor WSGI de Django\index{Django} por Gunicorn\index{Gunicorn}, o la detección de actualizaciones de dependencias con Dependabot\index{Dependabot}.
\end{itemize}
