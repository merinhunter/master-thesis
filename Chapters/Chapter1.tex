% Chapter 1: Introduction

\chapter{Introducción} % Main chapter title

\label{Chapter1} % For referencing the chapter elsewhere, use \ref{Chapter1}

%----------------------------------------------------------------------------------------

% Define some commands to keep the formatting separated from the content
\newcommand{\keyword}[1]{\textbf{#1}}
\newcommand{\tabhead}[1]{\textbf{#1}}
\newcommand{\code}[1]{\texttt{#1}}
\newcommand{\file}[1]{\texttt{\bfseries#1}}
\newcommand{\option}[1]{\texttt{\itshape#1}}

%----------------------------------------------------------------------------------------

\section{Motivación}

%----------------------------------------------------------------------------------------

\section{Estructura de la memoria}

La estructura que esta memoria sigue es la siguiente:

\begin{itemize}
    \item En el Capítulo~\ref{Chapter2} se explican varios conceptos y tecnologías que ya existen y que se han usado para llevar a cabo el proyecto.
    \item En el Capítulo~\ref{Chapter3} se detalla el desarrollo del proyecto, desde el trabajo previo al comienzo del proyecto hasta los problemas encontrados durante el desarrollo, pasando por los diferentes prototipos de la aplicación.
    \item En el Capítulo~\ref{Chapter4} se exponen las especificaciones del sistema final, incluyendo instrucciones detalladas para poner la aplicación en funcionamiento, así como un caso de uso.
    \item En el Capítulo~\ref{Chapter5} se finaliza la memoria haciendo una reflexión sobre los resultados y las posibles líneas de desarrollo futuras.
\end{itemize}

%----------------------------------------------------------------------------------------

\section{Objetivos}

\subsection{Objetivo general}

El objetivo principal de este proyecto es el desarrollo de una plataforma similar a Cauldron, que permita el análisis de ecosistemas \emph{open source}, modificando algunas partes de su funcionamiento original, añadiendo nuevas características, y resolviendo varios problemas que el proyecto original llevaba arrastrando desde su comienzo.

\subsection{Objetivos específicos}

Para llevar a cabo el objetivo principal del proyecto, se han elaborado una serie de objetivos más específicos:

\begin{itemize}
    \item
\end{itemize}
