% Chapter 2: State of the Art

\chapter{Estado del Arte} % Main chapter title

\label{Chapter2} % For referencing this chapter elsewhere, use \ref{Chapter2}

%-------------------------------------------------------------------------------

\section{Python\index{Python}}

Python\index{Python} es un lenguaje de programación orientada a objetos de alto nivel. Fue desarrollado originalmente por Guido van Rossum en el final de la década de 1980, y actualmente es gestionado por la \emph{Python Software Foundation} bajo una licencia \emph{open source}\index{open source} propia conocida como PSFL (\emph{Python Software Foundation License}). \emph{\parencite{Reference1}}

\begin{figure}[ht]
    \centering
    \includegraphics[width=0.3\textwidth]{Figures/python-logo}
    \decoRule
    \caption[Python (Logo)]{Logo de Python\index{Python} \emph{\parencite{Reference2}}}
    \label{fig:python-logo}
\end{figure}

Python\index{Python} es actualmente uno de los lenguajes de programación más populares. Esto se debe en gran parte a su modularidad, la cual permite agregar fácilmente nuevas piezas de código a aplicaciones ya desarrolladas. \emph{\parencite{Reference3}}

\subsection{Poetry}

Poetry\index{Poetry} es un gestor de paquetes y dependencias para Python\index{Python}. Entre sus características más destacadas se encuentra la generación de un fichero con la especificación exacta de dependencias (incluidas las dependencias de dependencias), lo que asegura instalaciones idénticas en diferentes entornos. \emph{\parencite{Reference6}}

Además de esto, Poetry\index{Poetry} incorpora la mayoría de reglas establecidas en PEP\footnote{PEP (\emph{Python Enhancement Proposal}) recoge y describe las principales guías y recomendaciones de la comunidad respecto a nuevas características y procesos para cualquier desarrollo con Python.}, como el uso del fichero \emph{pyproject.toml} para escribir la metainformación de los paquetes, o el uso del esquema \emph{major.minor.micro} para los identificadores de versión. \emph{\parencite{Reference6}}

%-------------------------------------------------------------------------------

\section{Django\index{Django}}

Django\index{Django} es un \emph{web framework} desarrollado para Python\index{Python} y que, al igual que este, cuenta con una licencia \emph{open source}\index{open source}. Fue desarrollado originalmente por Adrian Holovaty y Simon Willison, dos programadores web que trabajaban en un periódico. \emph{\parencite{Reference4}}

\begin{figure}[ht]
    \centering
    \includegraphics[width=0.3\textwidth]{Figures/django-logo}
    \decoRule
    \caption[Django (Logo)]{Logo de Django\index{Django} \emph{\parencite{Reference5}}}
    \label{fig:django-logo}
\end{figure}

Django\index{Django} sigue los principios de modularidad que caracterizan a Python\index{Python}, además de incluir muchas otras características interesantes como un sistema de vistas, modelos y plantillas que permite agilizar el desarrollo de aplicaciones web. Esto, sumado a un servidor web simple para desarrollos locales, una interfaz administrativa intuitiva, y la posibilidad de construir \emph{middleware} de todo tipo, convierten a Django\index{Django} en uno de los \emph{frameworks} para desarrollo web más famosos y usados de los últimos años. \emph{\parencite{Reference4}}

\subsection{Django REST Framework}

Django\index{Django} REST Framework (DRF) es un módulo para Django que amplía las características de este, y que facilita la creación de APIs REST manteniendo todas las ventajas de Django. \emph{\parencite{Reference7}}

\begin{figure}[ht]
    \centering
    \includegraphics[width=0.5\textwidth]{Figures/drf-logo}
    \decoRule
    \caption[DRF (Logo)]{Logo de DRF \emph{\parencite{Reference8}}}
    \label{fig:drf-logo}
\end{figure}

A diferencia del \emph{framework} original, DRF se centra principalmente en la parte \emph{backend} que compone una aplicación web. Para ello, ofrece al usuario muchas más opciones para serializar la entrada y salida de datos, además de ampliar las conocidas \emph{Generic Views} de Django con mejoras relacionadas con la construcción de APIs. \emph{\parencite{Reference7}}

Permite además interaccionar fácilmente con la API construida a través de una interfaz web básica, y es capaz de generar documentación dinámica para la API siguiendo la especificación de OpenAPI\footnote{\url{https://spec.openapis.org/oas/v3.1.0}}. \emph{\parencite{Reference7}}

\subsection{Gunicorn}

Gunicorn\index{Gunicorn} (\emph{"Green Unicorn"}) es un servidor HTTP\index{HTTP}, escrito en Python\index{Python}, que implementa WSGI\footnote{WSGI (\emph{Web Server Gateway Interface}) establece cómo deben ser llevadas las peticiones del servidor web a la aplicación web escrita en Python\emph{\parencite{Reference11}}.} como protocolo de llamadas.

Funciona utilizando un esquema centralizado, donde un nodo maestro gestiona y orquesta diferentes tipos de nodos trabajadores. Estos pueden ser utilizados para procesar peticiones síncronas o asíncronas, y su número escala en función de la demanda que tenga el servidor. \emph{\parencite{Reference9}}

\begin{figure}[ht]
    \centering
    \includegraphics[width=0.4\textwidth]{Figures/gunicorn-logo}
    \decoRule
    \caption[Gunicorn (Logo)]{Logo de Gunicorn \emph{\parencite{Reference10}}}
    \label{fig:gunicorn-logo}
\end{figure}

%-------------------------------------------------------------------------------

\section{GrimoireLab}

GrimoireLab\index{GrimoireLab} es el resultado de años de investigación y análisis de comunidades \emph{open source}\index{open source} y procesos de desarrollo. Desarrollado originalmente por miembros del equipo \emph{LibreSoft} (que más tarde fundarían Bitergia), actualmente forma parte de CHAOSS, una iniciativa de la \emph{Linux Foundation} centrada en crear y modelar métricas sobre comunidades \emph{open source}\index{open source}. \emph{\parencite{Reference12}}

\begin{figure}[ht]
    \centering
    \includegraphics[width=0.2\textwidth]{Figures/grimoirelab-logo}
    \decoRule
    \caption[GrimoireLab (Logo)]{Logo de GrimoireLab \emph{\parencite{Reference12}}}
    \label{fig:grimoirelab-logo}
\end{figure}

Se trata de un conjunto de herramientas diseñadas para recoger, enriquecer, consumir y visualizar datos de casi cualquier fuente relacionada con el desarrollo \emph{Open Source}\index{open source}.

Cada una de estas herramientas sirve a un propósito específico y, en su conjunto, proveen el servicio antes mencionado:

\begin{itemize}
    \item Perceval es el encargado de recoger los datos crudos de cada fuente de datos compatible con GrimoireLab\index{GrimoireLab}. Ya sea utilizando herramientas como Git\index{Git} o haciendo uso de la API\index{API} de la fuente analizada.
    \item GrimoireELK enriquece los datos recogidos por Perceval. Genera, a partir de estos, nuevos datos que son de interés para estudiar las comunidades \emph{open source}\index{open source}.
    \item Sorting Hat permite gestionar la información relacionada con identidades, incluso entre diferentes fuentes de datos.
    \item SirMordred es la herramienta orquestadora para GrimoireLab\index{GrimoireLab}. Se encarga de sincronizar a todas las componentes para dar servicio.
    \item ¡Y muchas más herramientas increíbles!
\end{itemize}

\begin{figure}[ht]
    \centering
    \includegraphics[width=0.7\textwidth]{Figures/grimoirelab-schema}
    \decoRule
    \caption[GrimoireLab (Esquema)]{Esquema de GrimoireLab \emph{\parencite{Reference12}}}
    \label{fig:grimoirelab-schema}
\end{figure}

\subsection{Cauldron.io}

Cauldron.io (o simplemente Cauldron\index{Cauldron}) es una plataforma web que permite a sus usuarios utilizar GrimoireLab\index{GrimoireLab} de manera sencilla y rápida, necesitando únicamente un navegador web para su uso. \emph{\parencite{Reference13}}

\begin{figure}[ht]
    \centering
    \includegraphics[width=0.2\textwidth]{Figures/cauldron-logo}
    \decoRule
    \caption[Cauldron (Logo)]{Logo de Cauldron \emph{\parencite{Reference13}}}
    \label{fig:cauldron-logo}
\end{figure}

Cauldron\index{Cauldron} nace como un SaaS\footnote{\emph{Software as a Service}} de GrimoireLab\index{GrimoireLab}, y actualmente tiene compatibilidad con algunas de las fuentes de datos más importantes dentro de los ecosistemas \emph{open source}\index{open source}, como GitHub, Meetup, o StackExchange. Está compuesto por una aplicación Django\index{Django} detrás de un servidor Nginx, y utiliza Open Distro for Elasticsearch\index{Elasticsearch} y Kibana\index{Kibana} para almacenar y visualizar los datos generados con GrimoireLab\index{GrimoireLab}, respectivamente.

\begin{figure}[ht]
    \centering
    \includegraphics[width=\textwidth]{Figures/cauldron-charts}
    \decoRule
    \caption[Cauldron (Gráficas)]{Gráficas en Cauldron \emph{\parencite{Reference13}}}
    \label{fig:cauldron-charts}
\end{figure}

Además de las visualizaciones generadas en Kibana\index{Kibana}, Cauldron\index{Cauldron} ofrece a sus usuarios otras formas de dar valor a sus análisis. Como se observa en la Figura~\ref{fig:cauldron-charts}, permite comparar métricas y gráficas de diferentes proyectos. Además de esto, permite generar gráficas personalizadas dentro de Kibana\index{Kibana} gracias a las funcionalidades extra de Open Distro for Elasticsearch\index{Elasticsearch}.

%-------------------------------------------------------------------------------

\section{FastAPI}

Sed ullamcorper quam eu nisl interdum at interdum enim egestas. Aliquam placerat justo sed lectus lobortis ut porta nisl porttitor. Vestibulum mi dolor, lacinia molestie gravida at, tempus vitae ligula. Donec eget quam sapien, in viverra eros. Donec pellentesque justo a massa fringilla non vestibulum metus vestibulum. Vestibulum in orci quis felis tempor lacinia. Vivamus ornare ultrices facilisis. Ut hendrerit volutpat vulputate. Morbi condimentum venenatis augue, id porta ipsum vulputate in. Curabitur luctus tempus justo. Vestibulum risus lectus, adipiscing nec condimentum quis, condimentum nec nisl. Aliquam dictum sagittis velit sed iaculis. Morbi tristique augue sit amet nulla pulvinar id facilisis ligula mollis. Nam elit libero, tincidunt ut aliquam at, molestie in quam. Aenean rhoncus vehicula hendrerit.

\subsection{Uvicorn}

Nunc posuere quam at lectus tristique eu ultrices augue venenatis. Vestibulum ante ipsum primis in faucibus orci luctus et ultrices posuere cubilia Curae; Aliquam erat volutpat. Vivamus sodales tortor eget quam adipiscing in vulputate ante ullamcorper. Sed eros ante, lacinia et sollicitudin et, aliquam sit amet augue. In hac habitasse platea dictumst.

%-------------------------------------------------------------------------------

\section{OpenSearch}

Sed ullamcorper quam eu nisl interdum at interdum enim egestas. Aliquam placerat justo sed lectus lobortis ut porta nisl porttitor. Vestibulum mi dolor, lacinia molestie gravida at, tempus vitae ligula. Donec eget quam sapien, in viverra eros. Donec pellentesque justo a massa fringilla non vestibulum metus vestibulum. Vestibulum in orci quis felis tempor lacinia. Vivamus ornare ultrices facilisis. Ut hendrerit volutpat vulputate. Morbi condimentum venenatis augue, id porta ipsum vulputate in. Curabitur luctus tempus justo. Vestibulum risus lectus, adipiscing nec condimentum quis, condimentum nec nisl. Aliquam dictum sagittis velit sed iaculis. Morbi tristique augue sit amet nulla pulvinar id facilisis ligula mollis. Nam elit libero, tincidunt ut aliquam at, molestie in quam. Aenean rhoncus vehicula hendrerit.

\subsection{Kibana}

Nunc posuere quam at lectus tristique eu ultrices augue venenatis. Vestibulum ante ipsum primis in faucibus orci luctus et ultrices posuere cubilia Curae; Aliquam erat volutpat. Vivamus sodales tortor eget quam adipiscing in vulputate ante ullamcorper. Sed eros ante, lacinia et sollicitudin et, aliquam sit amet augue. In hac habitasse platea dictumst.

%-------------------------------------------------------------------------------

\section{PostgreSQL}

Sed ullamcorper quam eu nisl interdum at interdum enim egestas. Aliquam placerat justo sed lectus lobortis ut porta nisl porttitor. Vestibulum mi dolor, lacinia molestie gravida at, tempus vitae ligula. Donec eget quam sapien, in viverra eros. Donec pellentesque justo a massa fringilla non vestibulum metus vestibulum. Vestibulum in orci quis felis tempor lacinia. Vivamus ornare ultrices facilisis. Ut hendrerit volutpat vulputate. Morbi condimentum venenatis augue, id porta ipsum vulputate in. Curabitur luctus tempus justo. Vestibulum risus lectus, adipiscing nec condimentum quis, condimentum nec nisl. Aliquam dictum sagittis velit sed iaculis. Morbi tristique augue sit amet nulla pulvinar id facilisis ligula mollis. Nam elit libero, tincidunt ut aliquam at, molestie in quam. Aenean rhoncus vehicula hendrerit.

%-------------------------------------------------------------------------------

\section{Docker}

Sed ullamcorper quam eu nisl interdum at interdum enim egestas. Aliquam placerat justo sed lectus lobortis ut porta nisl porttitor. Vestibulum mi dolor, lacinia molestie gravida at, tempus vitae ligula. Donec eget quam sapien, in viverra eros. Donec pellentesque justo a massa fringilla non vestibulum metus vestibulum. Vestibulum in orci quis felis tempor lacinia. Vivamus ornare ultrices facilisis. Ut hendrerit volutpat vulputate. Morbi condimentum venenatis augue, id porta ipsum vulputate in. Curabitur luctus tempus justo. Vestibulum risus lectus, adipiscing nec condimentum quis, condimentum nec nisl. Aliquam dictum sagittis velit sed iaculis. Morbi tristique augue sit amet nulla pulvinar id facilisis ligula mollis. Nam elit libero, tincidunt ut aliquam at, molestie in quam. Aenean rhoncus vehicula hendrerit.

%-------------------------------------------------------------------------------

\section{GitHub}

Sed ullamcorper quam eu nisl interdum at interdum enim egestas. Aliquam placerat justo sed lectus lobortis ut porta nisl porttitor. Vestibulum mi dolor, lacinia molestie gravida at, tempus vitae ligula. Donec eget quam sapien, in viverra eros. Donec pellentesque justo a massa fringilla non vestibulum metus vestibulum. Vestibulum in orci quis felis tempor lacinia. Vivamus ornare ultrices facilisis. Ut hendrerit volutpat vulputate. Morbi condimentum venenatis augue, id porta ipsum vulputate in. Curabitur luctus tempus justo. Vestibulum risus lectus, adipiscing nec condimentum quis, condimentum nec nisl. Aliquam dictum sagittis velit sed iaculis. Morbi tristique augue sit amet nulla pulvinar id facilisis ligula mollis. Nam elit libero, tincidunt ut aliquam at, molestie in quam. Aenean rhoncus vehicula hendrerit.

\subsection{Dependabot}

Nunc posuere quam at lectus tristique eu ultrices augue venenatis. Vestibulum ante ipsum primis in faucibus orci luctus et ultrices posuere cubilia Curae; Aliquam erat volutpat. Vivamus sodales tortor eget quam adipiscing in vulputate ante ullamcorper. Sed eros ante, lacinia et sollicitudin et, aliquam sit amet augue. In hac habitasse platea dictumst.

%-------------------------------------------------------------------------------

\section{HTTP}

Sed ullamcorper quam eu nisl interdum at interdum enim egestas. Aliquam placerat justo sed lectus lobortis ut porta nisl porttitor. Vestibulum mi dolor, lacinia molestie gravida at, tempus vitae ligula. Donec eget quam sapien, in viverra eros. Donec pellentesque justo a massa fringilla non vestibulum metus vestibulum. Vestibulum in orci quis felis tempor lacinia. Vivamus ornare ultrices facilisis. Ut hendrerit volutpat vulputate. Morbi condimentum venenatis augue, id porta ipsum vulputate in. Curabitur luctus tempus justo. Vestibulum risus lectus, adipiscing nec condimentum quis, condimentum nec nisl. Aliquam dictum sagittis velit sed iaculis. Morbi tristique augue sit amet nulla pulvinar id facilisis ligula mollis. Nam elit libero, tincidunt ut aliquam at, molestie in quam. Aenean rhoncus vehicula hendrerit.

\subsection{API REST}

Nunc posuere quam at lectus tristique eu ultrices augue venenatis. Vestibulum ante ipsum primis in faucibus orci luctus et ultrices posuere cubilia Curae; Aliquam erat volutpat. Vivamus sodales tortor eget quam adipiscing in vulputate ante ullamcorper. Sed eros ante, lacinia et sollicitudin et, aliquam sit amet augue. In hac habitasse platea dictumst.
